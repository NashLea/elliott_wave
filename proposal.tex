\documentclass[12pt]{article}
\usepackage{fullpage,enumitem,amsmath,amssymb,graphicx,bm,listings,algpseudocode,hyperref}
\newcommand{\vect}[1]{\boldsymbol{#1}}

\lstdefinestyle{custom}{
  basicstyle=\footnotesize\ttfamily,
  language=Python,
}
\begin{document}

\begin{center}
{\Large CS221 Fall 2016 Project [p-proposal]}

\begin{tabular}{rl}
SUNet ID: & motonari \\
Name: & Motonari ITO \\
Collaborators: & Shiva Sundar
\end{tabular}
\end{center}

\section{Scope}

\subsection{Elliott Wave Principle}

Elliott Wave Principle (EWP) is a hypothesis that stock market price
can be modeled as a sequence of waves which shapes follow some defined
rules. EWP suggests we can predict the future market price more
accurately than a random chance by determining the wave shape.

\subsection{Input and output behavior}

Given a history of a stock market, the system outputs the predicted
price change 30 days later. For example:

\begin{description}
\item[Input] AAPL stock data (Dec 12, 1980 - Oct 22, 2016), obtained from \url{https://finance.yahoo.com/quote/AAPL/history?p=AAPL}
\item[Output] \$123 (Predicted closing stock price of Nov 22, 2016)
\end{description}

\subsection{Evaluation metric to success}

Taking a history of some stock price, the evaluation process runs the
system against some past subrange of the period and compares the
output to the actual price. We use the mean squared error as the
evaluation metric.
      
\section{Baseline and Oracle}

\subsection{Baseline}

The baseline approach uses Dynamic Programming to find the wave
structure with simlified Elliott Wave rules below. Then, based on the
wave of today, it predicts the price after 30 days.

\begin{enumerate}
 \item There are five impulse waves (1, 2, 3, 4, and 5), followed by
   three corrective waves (A, B, and C).
 \item The input data (stock history) is a substring of the five waves.
 \item Each wave may have nested waves. Except the wave which ends at the end of the input data, these substructure must have all eight waves.
 \item Wave 2 never retraces more than 100\% of Wave 1.
 \item Wave 3 is longer than Wave 1 and 2. 
 \item Wave 4 does not enter into the same price territory as Wave 1.
\end{enumerate}

Running the baseline implementation against some stocks, we got the following baseline.

\begin{tabular}{cccr}
DJ & 2016-02-11 & 2016-06-27 & 446721.6  \\
AAPL & 2011-06-20 & 2012-09-17 & 410.8 \\
AAPL & 2013-06-24 & 2016-05-09 & 1110.9 \\
QDX  & 2006-10-02 & 2008-03-10 & 193.7 \\
RUT  & 2011-10-03 & 2016-02-08 & 51167.3 \\
\end{tabular}

\subsection{Oracle}

The oracle approach asks an expert opinion (Shiva Sunder) to label the
wave structure. Then, for each day in the history, it calculates the
mean squared error between the wave and the actual price. It gives the
accuracy of prediction given Elliott wave classification is perfectly
correct.

Oracle produced these values.

\begin{tabular}{cccr}
DJ \footnote{This is worse than baseline because the expert opinion doesn't have substructure. We need to correct this.} & 2016-02-11 & 2016-06-27 & 867453.8  \\
AAPL & 2011-06-20 & 2012-09-17 & 58.0 \\ 
AAPL & 2013-06-24 & 2016-05-09 & 358.5 \\
QDX  & 2006-10-02 & 2008-03-10 & 28.3 \\
RUT  & 2011-10-03 & 2016-02-08 & 26372.9 \\
\end{tabular}

\section{Challenges and Topics}

\begin{itemize}
\item Incorporate more Elliott rules into the dynamic programming and
  see if it increases the accuracy.
\item Define some mathematical curve of X nested level Elliott
  wave. Use linear regression to find the parameter to fit.
\item EWP is an old theory and can be out dated in the current stock
  behavior. We can try Reenforcement Learning to find the new rule
  (policy).
\end{itemize}

\section*{References}

\begin{itemize}
\item \url{https://en.wikipedia.org/wiki/Elliott_wave_principle}
\item \url{http://studyofcycles.com}
\end{itemize}
     
\end{document}




