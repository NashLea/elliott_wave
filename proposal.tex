\documentclass[12pt]{article}
\usepackage{fullpage,enumitem,amsmath,amssymb,graphicx,bm,listings,algpseudocode,hyperref}
\newcommand{\vect}[1]{\boldsymbol{#1}}

\lstdefinestyle{custom}{
  basicstyle=\footnotesize\ttfamily,
  language=Python,
}
\begin{document}

\begin{center}
{\Large CS221 Fall 2016 Project [p-proposal]}

\begin{tabular}{rl}
SUNet ID: & motonari \\
Name: & Motonari ITO \\
Collaborators: & Shiva Sundar
\end{tabular}
\end{center}

\section{Scope}

\subsection{Elliott Wave Principle}

Elliott Wave Principle (EWP) is a hypothesis that stock market price
can be modeled as a sequence of waves which shape follow some defined
rules.

By determining the wave shape of the day, EWP suggests we can predict
the future market price more accurately than a random chance.

\subsection{Input and output behavior}

Given a history of a stock market, the system outputs the predicted
price change 30 days later.

\begin{description}
\item[Input] AAPL stock data (Dec 12, 1980 - Oct 22, 2016), obtained from \url{https://finance.yahoo.com/quote/AAPL/history?p=AAPL}
\item[Output] \$123 (Predicted closing stock price of Nov 22, 2016)
\end{description}

\subsection{Evaluation metric to success}

Taking a history of some stock price, the evaluation process runs the
system against some past subrange of the period and compares the
output to the actual price.
      
\section{Baseline and Oracle}

\subsection{Baseline}

The baseline approach uses Dynamic Programming to find the wave
structure with simlified Elliott Wave rules. Then, based on the wave
of today, it predicts the price after 30 days.

Here is the simplified rules.

\begin{enumerate}
 \item There are five impulse waves (1, 2, 3, 4, and 5), followed by
   three corrective waves (A, B, and C).
 \item Wave 2 never retraces more than 100\% of Wave 1.
 \item Wave 3 is longer than Wave 1 and 2. 
 \item Wave 4 does not enter into the same price territory as Wave 1.
\end{enumerate}

[TODO] 

\subsection{Oracle}

The oracle approach asks an expert opinion (Shiva Sunder) to label the
wave structure of the full history. Then, for each day in the history,
it calculates the mean squared error between the wave and the actual
price. It gives the accuracy of prediction given Elliott wave
classification is correct.

[TODO] 

\section{Challenges and Topics}

\begin{itemize}
\item Incorporate more Elliott rules into the dynamic programming and
  see if it increases the accuracy.
\item Define some mathematical curve of X nested level Elliott
  wave. Use linear regression to find the parameter to fit.p
\item EWP is an old theory and can be out dated in the current stock
  behavior. We try Reenforcement Learning to find the new rule
  (policy).
\end{itemize}

\section*{References}

\begin{itemize}
\item \url{https://en.wikipedia.org/wiki/Elliott_wave_principle}
\item \url{http://studyofcycles.com}
\end{itemize}
     
\end{document}




