\documentclass[12pt]{article}
\usepackage{fullpage,enumitem,amsmath,amssymb,graphicx,bm,listings,algpseudocode,hyperref}
\newcommand{\vect}[1]{\boldsymbol{#1}}

\lstdefinestyle{custom}{
  basicstyle=\footnotesize\ttfamily,
  language=Python,
}
\begin{document}

\begin{center}
{\Large CS221 Fall 2016 Project [p-proposal]}

\begin{tabular}{rl}
SUNet ID: & motonari \\
Name: & Motonari ITO \\
Collaborators: & Shiva Sundar
\end{tabular}
\end{center}

\section*{Scope}

\subsection*{Elliott Wave Principle}

Elliott Wave Principle is a hypothesis that stock market price can be
modeled as a set of state and transition rules, and by determining the
current state, it suggests we could predict the future stock more
accurately than a random chance.

The state and transition rules are defined in terms of a sequence of
waves, namely wave 1, 2, 3, 4, 5, followed by wave A, B, and C as
illustrated below. Each wave may have this sequence as the
substructure, hence the nested structure. The hypothesis suggests that
the nested structure is applied all the way from yearly to hourly
changes.

Start and end time of each wave is determined by, but not limited to,
these rules. Note that, even if we follow all the rules strictly,
there can be multiple ways to associate a time period to a wave. 

[TODO] (Rules)


\subsection*{Input and output behavior}

[TODO] Given a history of a stock market, the system outputs the
predicted price change for tomorrow. \footnote{We could output the
  full wave structures as a nested time range table, but I feel this
  approach is easier to implement in various different algorithm.}

For example,

\begin{description}
\item[Input] AAPL stock data (Dec 12, 1980 - Oct 22, 2016), obtained from \url{https://finance.yahoo.com/quote/AAPL/history?p=AAPL}
\item[Output] Predicted closing stock price to be up or down.
\end{description}

\subsection*{Evaluation metric to success}

[TODO] some idea so far:

\begin{itemize}
\item Run the system on various stock, wait for tomorrow, and see if
  how it did. I think it's more fun. We can use contingency table to plot true-positive, false-positive, true-negative, and false-negative, and reports some metrics such as accuracy and F1-score.
\item Compare to some well-known stock prediction tool. What tool? 
\end{itemize}
      
\section*{Baseline and Oracle}

\subsection*{Baseline}

Use Dynamic Programming to find the wave structure with simlified
Elliott Wave rules. Based on the wave of today, predict if tomorrow's
price is going up or down.

\begin{enumerate}
 \item There are five impulse waves, followed by three corrective waves.
 \item Wave 2 doesn’t fall under (or upper if decreasing trend) the start of Wave 1.
 \item Wave 3 is the longest.
 \item Wave 4 doesn’t go in the territory of Wave 1.
 \item Each wave has a same substructure. For Baseline, we restrict the shortest wave to be 10 days. (?)
\end{enumerate}

\subsection*{Oracle}

[TODO] Maybe... how about taking some 30 days time period which is
known to be increasing wave. For example, AAPL's June 27, 2016 to Jul
20, 2016. Oracle predicts $+$ for all days. The ground truth is that
we have two bad days. So, Oracle accuracy is 28 / 30.

\section*{Challenges and Topics}

Some idea:

\begin{itemize}
\item Incorporate more Elliott rules into the dynamic programming and
  see if it increases the accuracy.
\item Define some mathematical curve of X nested level Elliott
  wave. Use linear regression to find the parameter to fit.p
\item You mentioned NN. I have no idea how to apply NN in this
  context. Could you elaborate?
\item Elliott is 100 years ago guy and his rule book may be out
  dated. Use RL to find the new rule (policy).
\end{itemize}

\section*{References}

\begin{itemize}
\item \url{https://en.wikipedia.org/wiki/Elliott_wave_principle}
\item \url{http://studyofcycles.com}
\end{itemize}
     
\end{document}




