\documentclass[12pt]{article}
\usepackage{fullpage,enumitem,amsmath,amssymb,graphicx,bm,listings,algpseudocode,hyperref}
\newcommand{\vect}[1]{\boldsymbol{#1}}

\lstdefinestyle{custom}{
  basicstyle=\footnotesize\ttfamily,
  language=Python,
}
\begin{document}

\begin{center}
{\Large CS221 Fall 2016 Project [p-proposal]}

\begin{tabular}{rl}
SUNet ID: & motonari \\
Name: & Motonari ITO \\
Collaborators: & Sundararaman Shiva
\end{tabular}
\end{center}

\section{Scope}

Elliott Wave Principle (EWP) is a hypothesis that stock market price
can be modeled as a sequence of waves which shapes follow some defined
rules. EWP suggests we can predict the future market price more
accurately than a random chance by recognizing the wave pattern.

This is distinct from other stock price prediction method in that it
relies sololy on the historical price changes and doesn't use external
information such as market sentiment or industrial news. While we
could improve the prediction by using those methods complementary, for
this project, we focus on EWP approach.

While EWP has several rules about wave shapes, we don't use them in
the original form (except in the baseline algorithm explored in
\verb|p-proposal|.) Instead, we use various AI algorithm to find a new
set of rules which predict the price more accurately.

\section{Model}

We model a stock market as an MDP where we neither know the
transitions nor reward functions. On a given day, we have prior stock
price history. We buy a set of stocks where each will be sold after
some days. T



\end{document}
